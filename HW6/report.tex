\documentclass[11pt]{article}
\usepackage{latexsym}
\usepackage{natbib}
\usepackage{graphicx}
\usepackage{caption}
\usepackage{subcaption}
\usepackage{listings}
\usepackage{algorithm}
\usepackage{algpseudocode}
\usepackage{hyperref}

\title{Homework 6: Genetic Algorithm}
\author{Shun Zhang}
\date{}

\begin{document}
\maketitle

\section{Genetic Algorithm for Sorting Networks}

For sorting network,

\begin{itemize}
\item Fitness:
$$Inversion(d) = I(i < j \land d[i] > d[j])$$
where I is an indicator function, which counts the number satisfied
pairs of $i, j$.
\item Selection:
\item Crossover:
\item Mutation:
\end{itemize}

For data to be sorted,

\begin{itemize}
\item Fitness:
\item Selection:
\item Crossover:
\item Mutation:
\end{itemize}

\section{Experiments}

\section{Discussion and Conclusion}

In my view, modular RL is one type of transfer learning. It apply the
results in some predefined source tasks to the target task. Also, some
learning could be allowed after combining the source tasks, rather
than our hand-tuning.

\end{document}
